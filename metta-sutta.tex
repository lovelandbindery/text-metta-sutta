\documentclass{book}
\usepackage{fontspec}
\usepackage{parskip}
\usepackage{relsize}
\usepackage{changepage}
\usepackage[
    paperheight=8.5in,
    paperwidth=5.5in,
    margin=1in
]{geometry}

\pagenumbering{gobble}
\newfontfamily\devanagari{Noto Sans Devanagari Regular}[Script=Devanagari]
\setmainfont{EB Garamond Medium}
\renewcommand\RSlargest{80pt}

\newcommand{\pali}[1]{\begin{adjustwidth}{0.6em}{0cm}\begin{LARGE}{\devanagari {#1} }\end{LARGE}\end{adjustwidth}}
\newcommand{\english}[1]{\begin{flushright}\begin{Large}{#1}\end{Large}\end{flushright}}
\newcommand{\firstletter}[1]{\relscale{2}{#1}}

\begin{document}

\newpage
\begin{center}
\begin{large}
{\devanagari
\relscale{1.75}{
करणीयमेत्ता सुत

}
}
\relscale{2}{
Karaṇīyamettā Sutta

}
\end{large}
\begin{Large}
\vspace{1.5cm}
A Discourse on Loving-Kindness\\

\end{Large}
\end{center}

\newpage
\null
\thispagestyle{empty}

\newpage

\pali{
करनियम् अत्थकुसलेन\\
यन् तम् सन्तम् पदम् अभिसमेच्च\\
सक्को उजु च सुजु च\\
सुवचो च्'अस्स मुदु अनतिमनि

}
\vfill
\english{
{\firstletter T}\kern-0.24em his is what should be done\\
By one who is skilled in goodness,\\
And who knows the path of peace:\\
Let them be able and upright,\\
Straightforward and gentle in speech,\\
Humble and not conceited,

}

\newpage
\pali{
सन्तुस्सको च सुभरो च\\
अप्पकिच्चो च सल्लहुकवुत्ति\\
सन्तिन्द्रियो च निपको च\\
अप्पगब्भो कुलेसु अननुगिद

}
\vfill
\english{
{\firstletter C}\kern-0.05em ontented and easily satisfied,\\
Unburdened with duties and frugal.\\
Peaceful and calm and wise and skillful,\\
Not proud or demanding in nature.

}

\newpage
\pali{
न च खुद्दम् समचरे किन्चि \\
येन विञ्ञु परे उपवदेय्युम् \\
सुखिनो व खेमिनो होन्तु \\
सब्बे सत्त भवन्तु सुखितत्त

}
\vfill
\english{
{\firstletter L}\kern+0.05em et them not do the slightest thing\\
That the wise would later reprove.\\
Let them think: In gladness and in safety,\\
May all beings be at ease.

}


\newpage
\pali{
ये केचि पनभुत्'अत्थि \\
तस व थवर व अनवसेस \\
दिघ व ये महन्त व \\
मज्झिम रस्सकनुकथुल

}
\vfill
\english{
{\firstletter W}\kern-0.25em hatever living beings there are,\\
Weak or strong,\\
Large, medium, or small,

}

\newpage
\pali{
दित्थ व येव अदित्थ \\
ये च दुरे वसन्ति अविदुरे \\
भुत व सम्भवेसि व \\
सब्बे सत्त भवन्तु सुखितत्त

}
\vfill
\english{
{\firstletter T}\kern-0.24em he seen and the unseen,\\
Those living near and far away,\\
Those born and yet to be born--\\
May all beings be at ease!

}

\newpage
\pali{
न परो परम् निकुब्बेथ \\
नतिमञ्ञेथ कत्थचिनम् कन्चि \\
ब्यरोसन पतिघसञ्ञ \\
नञ्ञमञ्ञस्स दुक्खम् इच्छेय्य

}
\vfill
\english{
{\firstletter L}\kern+0.05em et none decieve another,\\
Or despise any being in any state.\\
Let none through anger or ill-will\\
Wish harm upon another.

}

\newpage
\pali{
मत यथ नियम् पुत्तम् \\
अयुस एकपुत्तम् अनुरक्खे \\
एवम्पि सब्बभुतेसु \\
मनसम् भवये अपरिमनम्
}
\vfill
\english{
{\firstletter J}\kern-0.1em ust as a mother would protect\\
Her only child with her life\\
Even so let one cultivate\\
A boundless love towards all beings.

}

\newpage
\pali{
मेत्तञ् च सब्ब-लोकस्मिम् \\
मनसम् भवये अपरिमनम् \\
उद्धम् अधो च तिरियन्च \\
असम्बधम् अवेरम् असपत्तम्

}
\vfill
\english{
{\firstletter R}\kern+0.08em adiating kindness over the world:\\
Spreading upwards to the skies,\\
And downwards to the depths;\\
Outwards and unbounded,\\
Freed from hatred and ill-will.

}

\newpage
\pali{
तित्थञ् चरम् निसिन्नो व \\
सयनो व यवत्'अस्स विगतमिद्धो \\
एतम् सतिम् अधित्थेय्य \\
ब्रह्मम् एतम् विहरम् इधमहु

}
\vfill
\english{
{\firstletter W}\kern-0.25em hether standing, walking, seated,\\
Or lying down, as long as they are awake,\\
They should develop this mindfulness.\\
This they say, is the divine abiding here.

}

\newpage
\pali{
दित्थिञ्च अनुपगम्म सिलव \\
दस्सनेन सम्पन्नो \\
कमेसु विनेय्य गेधम् \\
न हि जतु गब्भसेय्यम् पुनर् एति'ति

}
\vfill
\english{
{\firstletter B}\kern+0.03em y not holding to fixed views,\\
The pure-hearted one,\\
Having clarity of vision,\\
With sensual desires abandoned,\\
Is not born again into this world.

}

\end{document}
